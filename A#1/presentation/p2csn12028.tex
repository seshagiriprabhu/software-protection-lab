% This text is Free and open Open Source.
% It's a part of presentation made by myself.
% It may be used only for academic purpose
% May, 2012
% Author: Seshagiri Prabhu
% Amrita Vishwa Vidyapeethm 
% seshagiriprabhu@gmail.com
% www.seshagiriprabhu.wordpress.com

\documentclass[12pt]{beamer}
\usetheme{Oxygen}
\usepackage{thumbpdf}
\usepackage{wasysym}
\usepackage{ucs}
\usepackage[utf8]{inputenc}
\usepackage{pgf,pgfarrows,pgfnodes,pgfautomata,pgfheaps,pgfshade}
\usepackage{verbatim}
\usepackage{listings}
\usepackage{courier}
\usepackage{caption}
\usepackage{verbatim} 
\usepackage{upquote}
\usepackage{graphics}
\usepackage{latexsym}
\usepackage{fixltx2e}
\usepackage{graphicx}
\usepackage{hyperref}
\usepackage{amssymb}
\usepackage{ragged2e}
\usepackage{amsmath}
\usepackage{mathtools}
\usepackage{pgf}
\usepackage{fmtcount}% http://ctan.org/pkg/fmtcount
\usepackage{algorithm, algpseudocode}
\usepackage{caption}
\captionsetup[algorithm]{font=scriptsize}
\usepackage{etoolbox}
\usepackage{minted}
\makeatother
\newcommand\Colorhref[3][cyan]{\href{#2}{\small\color{#1}#3}}
\newcommand\Fontvi{\fontsize{5}{6}\selectfont}
%\renewcommand\tinyv{\@setfontsize\tinyv{4pt}{6}}
%\renewcommand\tiny{\@setfontsize\tiny{4pt}{6}}
\usepackage{xcolor}
\def\SPSB#1#2{\rlap{\textsuperscript{\textcolor{red}{#1}}}\SB{#2}}
\def\SP#1{\textsuperscript{\textcolor{red}{#1}}}
\def\SB#1{\textsubscript{\textcolor{blue}{#1}}}

\pdfinfo
{
  /Title       (SN 707 Software Protection)
  /Creator     (Seshagiri Prabhu N)
  /Author      (TeX)
}
\title{SN 707 Software Protection}
\subtitle{Assignment 1}
\author{Seshagiri Prabhu N}
\institute[Amrita Vishwa Vidyapeetham] % (optional)
{
  \begin{center}
    \tiny \textbf{AM.EN.P2CSN12028} \\ 
    M.Tech, Cyber Security and Networks\\
  	Amrita School of Engineering,
	Amritapuri Campus
  \end{center}  
}

\begin{document}
\frame{\titlepage}

\begin{frame}[fragile]{Problem}
	\begin{minted}[C]
		#include <stdio.h>
		#define _ -F<00||--F-OO--;
		int F=00,OO=00;main(){F_OO();printf("%1.3f\n",4.*-F/OO/OO);}F_OO()
		{
		            _-_-_-_
	    	   _-_-_-_-_-_-_-_-_
		   	 _-_-_-_-_-_-_-_-_-_-_-_
		  _-_-_-_-_-_-_-_-_-_-_-_-_-_
		 _-_-_-_-_-_-_-_-_-_-_-_-_-_-_
		 _-_-_-_-_-_-_-_-_-_-_-_-_-_-_
		_-_-_-_-_-_-_-_-_-_-_-_-_-_-_-_
		_-_-_-_-_-_-_-_-_-_-_-_-_-_-_-_
		_-_-_-_-_-_-_-_-_-_-_-_-_-_-_-_
		_-_-_-_-_-_-_-_-_-_-_-_-_-_-_-_
		 _-_-_-_-_-_-_-_-_-_-_-_-_-_-_
		 _-_-_-_-_-_-_-_-_-_-_-_-_-_-_
		  _-_-_-_-_-_-_-_-_-_-_-_-_-_
		    _-_-_-_-_-_-_-_-_-_-_-_
		        _-_-_-_-_-_-_-_
		            _-_-_-_
		}
	\end{minted} 	
\end{frame}

\section*{}
\begin{frame}
  \frametitle{Outline}
  \tableofcontents[section=1,hidesubsections]
\end{frame}

\newcommand{\icon}[1]{\pgfimage[height=1em]{#1}}

%%%%%%%%%%%%%%%%%%%%%%%%%%%%%%%%%%%%%%%%%
%%%%%%%%%% Content starts here %%%%%%%%%%
%%%%%%%%%%%%%%%%%%%%%%%%%%%%%%%%%%%%%%%%%

\section{Introduction}

\begin{frame}{timebomb}
  \frametitle{Introduction}
	\begin{enumerate}
		\justifying
		\item Relevance of \textbf{Modulo arithmetic} in public key crypto system
		\item \textbf{Euclid}'s algorithm, is an efficient method for computing the greatest common divisor (GCD) of two integers
		\item The use of \textbf{E}xtended \textbf{E}uclidean \textbf{A}lgorithm (\textbf{EEA}) to evaluate the multiplicative inverse
	\end{enumerate}
\end{frame}


\section{Contribution of this paper}
\begin{frame}
\frametitle{Contributions of papers}
	\begin{block}{"Contributions" of the research paper 1}
	\justifying
		Computerized algorithm (the first, efficient and cost effective) for the determination of the multiplicative inverse of a polynomial over GF(2\SP{m}) using simple bit wise shift and XOR operations.
	\end{block}
	\begin{block}{Contributions of the research paper 2}
		\justifying
		An algorithm for inversion in $GF(2^{m})$ suitable for implementation using a polynomial multiply instruction on $GF(2)$.
	\end{block}
\end{frame}

\section{Problem Description}
\begin{frame}
\frametitle{Problem Description}
\framesubtitle {Euclid Algorithm}
	\begin{block}{Euclid Algorithm}
		\scriptsize
		Euclid’s algorithm for polynomial calculates the greatest common divisor (GCD) polynomial of two polynomials
	\end{block}
	
	\begin{block}{EA description}
		\tiny
		The Euclidean algorithm proceeds in a series of steps such that the output of each step is used as an input for the next one.\\
		$r_{k-2} = q_{k} \times r_{k-1} + r_{k}$\\
		Where  $deg(r_{k}) < deg (r_{k-1})$ and $\forall k \geq 0$. \\
		If $r_{k-2} < r_{k-1}$ then we $r_{k-2} \rightleftharpoons r_{k-1}$
	\end{block}
	
	\begin{block}{Example: a = 1071 and b = 462}
		\begin{center}
		\tiny
			\begin{tabular}{| c | c | c | c |}
				\hline
				k & Equation & $q_{k}$  & $r_{k}$ \\ \hline
				0 & $1071  = q_{0}\times462 + r_{0}$ & 2 & 147 \\ \hline
				1 & $462  = q_{1}\times147 + r_{1}$ & 3 & 21 \\ \hline
				2 & $147  = q_{2}\times21 + r_{2}$ & 7 & 0  \\ \hline
			\end{tabular}
		\end{center}
	\end{block}
\end{frame}

\begin{frame}
	\frametitle{Problem Description}
	\framesubtitle{Extended Euclidean Algorithm}
	\begin{block}{EEA}
		Let \textcolor{red}{$A$} and \textcolor{red}{$B$} be polynomials.\\
		EEA gives \textcolor{red}{$U$} and \textcolor{red}{$V$} such that\\
		$\gcd{(A, B)} = U \times A + V \times B$\\
	\end{block}
		\begin{block}{Polynomial representation}
	  	The finite field is a representative of a polynomial function with respect to one variable x: \\
	  	$GF(2\SP{p}) = c_{p-1}x\SP{p-1} + c_{p-2}x\SP{p-2} + \dots + c_{2}x\SP{2} + c_{1}x\SP{1} + c_{0}$ \\
	  	$\mid c_{i} \in F_{2} =$ \{0,1\}
	\end{block}
	
\end{frame}
\begin{frame}
\frametitle{Problem Description}
	\begin{block}{Modular multiplicative inverse}
	\justify
		If \textcolor{red}{$A$} and \textcolor{red}{$B$} are irreducible, then $\gcd(A, B)$ is \textcolor{red}{$1$}.\\
		The modular multiplicative inverse of  modulo b is U \\ $\mid$ \hspace{2mm} $A^{-1} \equiv U\hspace{2mm} [mod\hspace{2mm}B]$
	\end{block}
\end{frame}

\begin{frame}
\frametitle{Example}
\scriptsize
	\begin{block}{Problem}
		We have $G = x\SP{8} + x\SP{4} + x\SP{3} + x + 1$ \\
		Let $ P = x^{6} + x^{4} + x + 1$ and $Q = x^7 + x^6 + x^3 + x$\\
		$P \times Q$ = $x^{13} + x^{12} + x^{11} + x^{10} + x^{9} + x^{8} +  4 * x^{7} + x^{6} + x^{5} + x^{4}  + x^{3} + x^{2} + x$\\
		$P \times Q$ = $x^{13} + x^{12} + x^{11} + x^{10} + x^{9} + x^{8} + x^{6} + x^{5} + x^{4}  + x^{3} + x^{2} + x$\\
	\end{block}		
	\begin{block}{Computations}
	\begin{tabular}{|c|c|}
			\hline
			$P \times Q + (x^{5})\times G$ & $x^{12} + x^{11} + x^{10} + x^{4} + x^{3} + x$ \\ \hline
			$P \times Q + (x^{5} + x^{4})\times G$ & $x^{11} + x^{10} + x^{3} + x$ \\ \hline
			$P \times Q + (x^{5} + x^{4} + x^3)\times G$ & $x^{10} + x$ \\		\hline
			$P \times Q + (x^{5} + x^{4} + x^{3} + x^{2})\times G$ &  $x$ \\		\hline	
			$P \times Q + (x^{5} + x^{4} + x^{3} + x^{2} + 1)\times G$ &  $1$ \\	\hline
	\end{tabular}					
	\end{block}

	\begin{block}{Result}
		The M.I of PQ on GF(2\SP{8}) is x\SP{5} + x\SP{4} + x\SP{3} + x\SP{2} + 1		
	\end{block}
\end{frame}

\section{Traditional Algorithm}
\begin{frame}[fragile]
\frametitle{Traditional Algorithm}
\begin{algorithm}[H]
		\label{euclid}
		\begin{algorithmic}
			\scriptsize
			\Procedure{Extended Euclid's Algorithm}{$A_{3}[\hspace{1mm}]$, $B_{3}[\hspace{1mm}]$}
				\State $R_{-1}(x) := B(x); U_{-1}(x) := 0; W_{-1}(x) := 1;$
				\State $R_{0}(x) := A(x); U_{0}(x) := 1; W_{0}(x) := 0;$
				\State $j := 0;$ 
				\While {($R_{j}(x) \neq 0$)} 
					\State $j := j + 1$;
					\State $Q_{j}(x) = R_{j-2}(x) \div R_{j-1}(x);$
					\State $R_{j}(x) = R_{j-2}(x)  - Q_{j}(x) \times R_{j-1}(x);$
					\State $U_{j}(x) = U_{j-2}(x)  - Q_{j}(x) \times U_{j-1}(x);$
					\State $W_{j}(x) = W_{j-2}(x)  - Q_{j}(x) \times W_{j-1}(x);$					
				\EndWhile
			\EndProcedure
		\end{algorithmic}
		\caption{Extended Euclidean Algorithm}
	\end{algorithm}
\end{frame}

\begin{frame}
\frametitle{Example} 
\tiny
Let's apply EEA to B = 203  and A = 80
	\begin{center}
	\tiny
		\begin{tabular}{| l | p{4cm} | c | r | r | r |}
			\hline
			j & Operation & $R_{j}$ & $U_{j}$ & $W_{j}$ & $Q_{j}$ \\ \hline
			-1 & $R_{-1}(x) := B(x); \newline U_{-1}(x) := 0; W_{-1}(x) := 1$ & $11001011$ & 0 & 1 & - \\ \hline 
			0 & $R_{0}(x) := A(x); \newline U_{0}(x) := 1;  W_{0}(x) := 0;$ & $01010000$ & 1 & 0 & -\\ \hline
			1 & $Q_{1}(x) := R_{-1}(x) \div R_{0}(x)$ & - & - &  - & $0011$\\ 
			  & $R_{1}(x) := R_{-1}(x) - Q_{1}(x) \times R_{0}(x)$ & $00111011$& - & - & - \\
			  & $U_{1}(x) := U_{-1}(x) - Q_{1}(x) \times U_{0}(x)$ & - & $0011$ & - & -\\
			  & $W_{1}(x) := W_{-1}(x) - Q_{1}(x) \times W_{0}(x)$ & - & - & 1 & - \\ \hline
			2 & $Q_{2}(x) := R_{0}(x) \div R_{1}(x)$ & - & - &  - & $0011$\\ 
			  & $R_{2}(x) := R_{0}(x) - Q_{2}(x) \times R_{1}(x)$ & $00011101$& - & - & - \\
			  & $U_{2}(x) := U_{0}(x) - Q_{2}(x) \times U_{1}(x)$ & - & $100$ & - & -\\
			  & $W_{2}(x) := W_{0}(x) - Q_{2}(x) \times W_{1}(x)$ & - & - & $11$ & - \\ \hline			
 			3 & $Q_{3}(x) := R_{1}(x) \div R_{2}(x)$ & - & - &  - & $0010$\\ 
			  & $R_{3}(x) := R_{1}(x) - Q_{3}(x) \times R_{2}(x)$ & $1$& - & - & - \\
			  & $U_{3}(x) := U_{1}(x) - Q_{3}(x) \times U_{2}(x)$ & - & $\underline{1011}$ & - & -\\
			  & $W_{3}(x) := W_{1}(x) - Q_{3}(x) \times W_{2}(x)$ & - & - & $111$ & - \\ \hline	
		\end{tabular}
	\end{center}

\end{frame}

\section{Proposed Algorithms}
\begin{frame}[fragile]
	\frametitle{Proposed Algorithms}
	\framesubtitle{Research Paper 1}
	\begin{algorithm}[H]
		%\algsetup{linenosize=\tiny}
	  	\tiny
		\label{euclid}
		\begin{algorithmic}
			\Procedure{Multiplicative Inverse}{$A_{3}[\hspace{1mm}]$, $B_{3}[\hspace{1mm}]$}
				\State $C\SB{1} = A\SB{2} = B\SB{2} = 0$;
				\While {($B\SB{3} \textgreater 1$)} \Comment{Step 1 do}
					\State $Q = 0$;
					\State $Temp = B\SB{3}$;
					\While {($A\SB{}3 > Temp \parallel BitSize(C) \geq BitSize(Temp)$)}
						\State $Q\SB{1} = 1$;
						\While {($A\SB{3MSB} == B\SB{3MSB}$)}
							\State $B\SB{3} = B\SB{3} << LinearLeftShift$ ;
							\State $Q\SB{1} = Q\SB{1} * 2$;
						\EndWhile
							\State $Q = Q + Q\SB{1}$;
							\State $A\SB{3} = A\SB{3}[\hspace{1mm}] \oplus B\SB{3}[\hspace{1mm}]$;
							\State $B\SB{3} = Temp$;
					\EndWhile
					\State $A\SB{2} = B\SB{2}; B\SB{3} = A\SB{3}; A\SB{3} = Temp$;
					\State $N = BitSize(Q)$; \Comment {Binary Bit Size of Q}
					\State $Temp = B\SB{2}; C\SB{2}=0;$\Comment {Step2}	
					\While{($N\geq1$)}
						\State $C\SB{2} = 0\SB{d}$;
						\If {($Q\SB{N}==1$)}\Comment {Testing if Nth bit of Q is 1}
							\State $C\SB{1}=B\SB{2}<< N-1$; \Comment{Linear left shift by N-1 times}
							\State $C\SB{2} = C\SB{2}\oplus C\SB{1}$;
						\EndIf
						\State $N--$;
					\EndWhile
					\State $B\SB{2} = C\SB{2}; A\SB{2} = Temp; B\SB{2}=B\SB{2} \oplus A\SB{2};$ \Comment  {Multiplicative Inverse}
				\EndWhile
			\EndProcedure
		\end{algorithmic}
%		\caption{\tiny {Euclidean Algorithm}}
	\end{algorithm}
\end{frame}

\begin{frame}
	\frametitle{Example}
	\framesubtitle{Let's apply EEA to A = 283 and B = 42}
	\tiny 
	\begin{center}
		\begin{tabular}{| l | c | c | r | r |}
			\hline
			i & Operation & Binary & V & U \\ \hline
			0 & $A$ & $100011011$ & 1 & 0 \\ \hline 
			1 & $B$ & $000101010$ & 0 & 1 \\ \hline
			  & $3<<B$ & $101010000$ & 0 & 1000 \\ 
			2 & $A \leftarrow A \oplus (3<<B)$ & $001001011$ & 1 & 1000 \\ \hline
			 & $1<<B$ & $001010100$ & 0 & 0010 \\
			3 & $A \leftarrow A \oplus (1<<B)$ & $000011111$ & 1 & 1010 \\ \hline
			 & $A<B \hspace{1mm} A \rightleftharpoons B$ & & & \\
			 & $A$ & $000101010$ & 00 & 00001 \\
			 & $B$ & $000011111$ & 01 & 01010 \\
			 & $1<<B$ & $000111110$ & 10 & 10100 \\
			4 & $A \leftarrow A \oplus (1<<B)$ & $000010100$ & 10 & 10101 \\ \hline
			 & $A<B \hspace{1mm} A \rightleftharpoons B$ & & & \\
			 & $A$ & $000011111$ & 01 & 01010 \\
			 & $B$ & $000010100$ & 10 & 10101 \\
			5 & $A \leftarrow A \oplus B$ & $000001011$ & 11 & 11111 \\ \hline
			 & $A<B \hspace{1mm} A \rightleftharpoons B$ & & & \\
			 & $A$ & $000010100$ & 010 & 010101 \\
			 & $B$ & $000001011$ & 011 & 011111 \\
			 & $1<<B$ & $000010110$ & 110 & 111110 \\
			6 & $A \leftarrow A \oplus (1<<B)$ & $000000010$ & 100 &  101011 \\ \hline	
			 & $A<B \hspace{1mm} A \rightleftharpoons B$ & & & \\
			 & $A$ & $000001011$ & 00011 & 00011111 \\
			 & $B$ & $000000010$ & 00100 & 00101011 \\
			 & $2<<B)$ & $000001000$ & 10000 & 10101100 \\
			7 & $A \leftarrow A \oplus (2<<B)$ & $000000011$ & 10011 & 10110011 \\ \hline
			8 & $A \leftarrow A \oplus B$ & $000000001$ & 10111 & $\underline{10011000}$ \\ \hline		 
		\end{tabular}
	\end{center}
\end{frame}

\begin{frame}
\frametitle{Proposed Algorithms}
\framesubtitle{Research Paper 2}
\begin{algorithm}[H]
\tiny
\label{euclid}
	\begin{algorithmic}
		\Procedure{Extended Euclid's Algorithm}{$A_{3}[\hspace{1mm}]$, $B_{3}[\hspace{1mm}]$}
			\State $S(x) := G(x); V(x) := 0;$
			\State $R(x) := A(x); U(x) := 1;$
			\While {$(deg(R(x)\not=0)$} 
				\State $\delta := deg(S(x)) - deg(R(x));$
				\If {$\delta < 0$}
					\State $S(x)\rightleftharpoons R(x)$
					\State $V(x) \rightleftharpoons U(x)$
					\State $\delta := -\delta;$				
				\EndIf
				\State $S(x) := S(x) - x^{\delta} \times R(x);$
				\State $V(x) := V(x) - x^{\delta} \times U(x);$
			\EndWhile
		\EndProcedure
	\end{algorithmic}
	\caption{ Algorithm for Inversion in $GF(2^{m})$ Based on EEA}
\end{algorithm}
\scriptsize
Output $U(x)$ as the result\\
$(U(x) = A^{-1}(x))$
\end{frame}

\begin{frame}
\frametitle{Example}
\framesubtitle {$G(2^{7}) = x^{7}+x^{6}+x^{3}+x+1, A(x) = x^{6}+x^{4}$}
\tiny
	\begin{center}
	\tiny
		\begin{tabular}{| p{3cm} | c | r | r | r | r |}
			\hline
			Operation	 & $R(x)$ & $S(x)$ & $U(x)$ & $V(X)$ & $\delta$ \\ \hline
			$S(x) := G(x); V(x) :=0; \newline R(x) := A(x); U(x) = 1;$ & $1010000$ & $11001011$ & 1 & 0 & \\ \hline 
			$\delta := deg(S(x)) - deg(R(x)); \newline S(x) := S(x) - x^{\delta} \times R(x);\newline V(x) := V(x) - x^{\delta} \times U(x);$ & $1010000$ & $1101011$ & 1 & 10 & 1\\ \hline 
			$\delta := deg(S(x)) - deg(R(x)); \newline  S(x) := S(x) - x^{\delta} \times R(x);\newline V(x) := V(x) - x^{\delta} \times U(x);$ & $1010000$ & $111011$ & 1 & 11 & 0\\ \hline
			$\delta := deg(S(x)) - deg(R(x)); \newline S(x) \rightleftharpoons R(x); \newline V(x) \rightleftharpoons U(x));  \delta := -\delta; \newline S(x) := S(x) - x^{\delta} \times R(x);\newline V(x) := V(x) - x^{\delta} \times U(x);$& $111011$ & $100110$ & 11 & 111 & $1$\\ \hline
			$\delta := deg(S(x)) - deg(R(x)); \newline S(x) := S(x) - x^{\delta} \times R(x);\newline V(x) := V(x) - x^{\delta} \times U(x);$ & $111011$ & $11101$ & 1 & 100 & 0\\ \hline 
			$\delta := deg(S(x)) - deg(R(x)); \newline S(x) \rightleftharpoons R(x); \newline V(x) \rightleftharpoons U(x)); \delta := -\delta; \newline S(x) := S(x) - x^{\delta} \times R(x);\newline V(x) := V(x) - x^{\delta} \times U(x);$& $11101$ & $1$ & 111 & 1011 & $1$\\ \hline			
			$\delta := deg(S(x)) - deg(R(x)); \newline S(x) \rightleftharpoons R(x); \newline V(x) \rightleftharpoons U(x)); \delta := -\delta; \newline S(x) := S(x) - x^{\delta} \times R(x);\newline V(x) := V(x) - x^{\delta} \times U(x);$& $1$ & $1101$ & $\underline{1011}$ & 1111011 & $4$\\ \hline			
		\end{tabular}
	\end{center}
\end{frame}

\section{Conclusion}
\begin{frame}
\frametitle{Conclusion}
	\begin{enumerate}
		\item The $algorithm^{[1]}$ is the pseudo-code of the $algorithm^{[2]}$.
	   	\item The $Algorithm^{[1]}$ is badly written such that it requires reverse engineering to understand the algorithm.
	   	\item $Algorithm^{[2]}$ is easy to understand and implement.
	   	\item $Algorithm^{[1]}$ doesn't store $V$, but it caches a lot of temporary variables.
	   	\item $Paper^{[1]}$ (published in 2009) claims that it is a pioneering work and the results could not be compared with that of any previous work.
	\end{enumerate}
\end{frame}

\begin{frame}
\frametitle{Future works}
  \begin{block}{Possible future works}
  	\begin{enumerate}
  		\item Comparative study with the latest efficient EEAs
  		\item Further optimize the EEA
  		\item Implementation in hardware for real time applications (VLSI algo)
  	\end{enumerate}
  \end{block}
\end{frame}

\begin{frame}
\frametitle {Coding!}
	\begin{block}{Implemented}
		\begin{enumerate}
			\item Binary Vectors
			\item Boolean function
			\item Walsh Transform
			\item Fast Walsh Transform
			\item Algebraic Normal form (ANF)
		\end{enumerate}
	\end{block}

	\begin{block}{Source code}
	The source code is available here: \Colorhref{https://github.com/seshagiriprabhu/case-study-1}		 {https://github.com/seshagiriprabhu/case-study-1}

	\end{block}
\end{frame}

\frame{
  \vspace{2cm}
  {\huge Questions ?}
  \vspace{3cm}
  \begin{flushright}
	Seshagiri Prabhu \\
    \structure{\footnotesize{seshagiriprabhu@gmail.com}}
  \end{flushright}
}

\begin{frame}
\frametitle{References}
References: \newline
		[1] \Colorhref{http://ijns.femto.com.tw/contents/ijns-v10-n2/ijns-2010-v10-n2-p107-113.pdf}{http://ijns.femto.com.tw/contents/ijns-v10-n2/ijns-2010-v10-n2-p107-113.pdf} \newline
		[2] \Colorhref{http://www2.lirmm.fr/arith18/papers/kobayashi-AlgorithmInversionUsingPolynomialMultiplyInstruction.pdf}{http://www2.lirmm.fr/arith18/papers/kobayashi-AlgorithmInversionUsingPolynomialMultiplyInstruction.pdf}
\end{frame}
\end{document}
