% This text is Free and open Open Source.
% It's a part of presentation made by myself.
% It may be used only for academic purpose
% May, 2012
% Author: Seshagiri Prabhu
% Amrita Vishwa Vidyapeethm 
% seshagiriprabhu@gmail.com
% www.seshagiriprabhu.wordpress.com

\documentclass[12pt]{beamer}
\usetheme{Oxygen}
\usepackage{thumbpdf}
\usepackage{wasysym}
\usepackage{ucs}
\usepackage[utf8]{inputenc}
\usepackage{pgf,pgfarrows,pgfnodes,pgfautomata,pgfheaps,pgfshade}
\usepackage{verbatim}
\usepackage{listings}
\usepackage{courier}
\usepackage{caption}
\usepackage{verbatim} 
\usepackage{upquote}
\usepackage{graphics}
\usepackage{latexsym}
\usepackage{fixltx2e}
\usepackage{graphicx}
\usepackage{hyperref}
\usepackage{amssymb}
\usepackage{ragged2e}
\usepackage{amsmath}
\usepackage{mathtools}
\usepackage{framed,lipsum}
\usepackage{pgf}
\usepackage{fmtcount}% http://ctan.org/pkg/fmtcount
\usepackage{algorithm, algpseudocode}
\usepackage{caption}
\captionsetup[algorithm]{font=scriptsize}
\usepackage{etoolbox}
\usepackage{xcolor}
\makeatother
\newcommand\Colorhref[3][cyan]{\href{#2}{\small\color{#1}#3}}
\newcommand\Fontvi{\fontsize{5}{6}\selectfont}
%\renewcommand\tinyv{\@setfontsize\tinyv{4pt}{6}}
%\renewcommand\tiny{\@setfontsize\tiny{4pt}{6}}
\usepackage{xcolor}
\setbeamertemplate{itemize items}[default]
\setbeamertemplate{enumerate items}[default]
\def\SPSB#1#2{\rlap{\textsuperscript{\textcolor{red}{#1}}}\SB{#2}}
\def\SP#1{\textsuperscript{\textcolor{red}{#1}}}
\def\SB#1{\textsubscript{\textcolor{blue}{#1}}}

\usepackage{tikz}
\usetikzlibrary{decorations}
\usepackage{textcomp} 
\usetikzlibrary{snakes}
\usepackage{ifthen}
\usepackage[firstyear=2008,lastyear=2011]{moderntimeline}
\colorlet{color0}{blue}
\colorlet{color1}{olive}

\newcommand*{\hintfont}{}
\newcommand*{\hintstyle}[1]{{\hintfont\textcolor{color0}{#1}}}
\newcommand*{\listitemsymbol}{a~}
\newcommand*{\cventry}[7][.25em]{%
  \cvitem[#1]{#2}{%
    {\bfseries\raggedright #3}%
    \ifthenelse{\equal{#4}{}}{}{, \raggedright{\slshape#4}}%
    \ifthenelse{\equal{#5}{}}{}{,  \raggedright#5}%
    \ifthenelse{\equal{#6}{}}{}{, \raggedright#6}%
    .\strut%
    \ifx&#7&%
      \else{\newline{}\begin{minipage}[t]{\linewidth}\small\raggedright#7\end{minipage}}\fi}}
\newcommand*{\cvitem}[3][.25em]{%
  \begin{tabular}{@{}p{\hintscolumnwidth}@{\hspace{\separatorcolumnwidth}}p{\maincolumnwidth}@{}}%
      \raggedleft\hintstyle{#2} & {#3}%
  \end{tabular}%
  \par\addvspace{#1}}
\tlmaxdates{2008}{2011}
\newlength{\quotewidth}
\newlength{\hintscolumnwidth}
\setlength{\hintscolumnwidth}{0.175\textwidth}
\newlength{\separatorcolumnwidth}
\setlength{\separatorcolumnwidth}{0.025\textwidth}
\newlength{\maincolumnwidth}
\newlength{\doubleitemmaincolumnwidth}
\newlength{\listitemsymbolwidth}
\settowidth{\listitemsymbolwidth}{\listitemsymbol}
\newlength{\listitemmaincolumnwidth}
\newlength{\listdoubleitemmaincolumnwidth}
\setlength{\maincolumnwidth}{\dimexpr0.9\linewidth-\separatorcolumnwidth-\hintscolumnwidth\relax}
\makeatother

\newenvironment{variableblock}[3]{%
  \setbeamercolor{block body}{#2}
  \setbeamercolor{block title}{#3}
  \begin{block}{#1}}{\end{block}}

\usepackage{listings}
\usepackage{color}

\definecolor{mygreen}{rgb}{0,0.6,0}
\definecolor{mygray}{rgb}{0.5,0.5,0.5}
\definecolor{mymauve}{rgb}{0.58,0,0.82}
\lstset{
 breaklines=true, commentstyle=\color{mygreen}, stepnumber=1, tabsize=2, stringstyle=\color{mymauve}, numberstyle=\tiny\color{mygray}, rulecolor=\color{black}, morekeywords={make,mkdir,gcc, all, clean, F, OO}, basicstyle=\tiny\ttfamily,keywordstyle=\color{blue}, emph={CC, CFLAG, EXECS, PROG, OBJS, OBJECTS, PHONY}, emphstyle=\color{red}, emph={[2]\#using,\#define,\#ifdef,\#endif}, emphstyle={[2]\color{blue}}, stringstyle=\color{red}
}
\pdfinfo
{
  /Title       (SN 707 Software Protection)
  /Creator     (Seshagiri Prabhu N)
  /Author      (TeX)
}
\title{SN 707 Software Protection}
\subtitle{Assignment 2}
\author{Seshagiri Prabhu N}
\institute[Amrita Vishwa Vidyapeetham] % (optional)
{
  \begin{center}
    \tiny \textbf{AM.EN.P2CSN12028} \\ 
    M.Tech, Cyber Security and Networks\\
  	Amrita School of Engineering,
	Amritapuri Campus
  \end{center}  
}

\begin{document}
\frame{\titlepage}


\begin{frame}
  \frametitle{Outline}
  \tableofcontents[section=1,hidesubsections]
\end{frame}

\newcommand{\icon}[1]{\pgfimage[height=1em]{#1}}

%%%%%%%%%%%%%%%%%%%%%%%%%%%%%%%%%%%%%%%%%
%%%%%%%%%% Content starts here %%%%%%%%%%
%%%%%%%%%%%%%%%%%%%%%%%%%%%%%%%%%%%%%%%%%

\section{Invention}
\begin{frame}
	\frametitle{Invention}
	\begin{variableblock}{}{bg=orange,fg=black}{bg=green,fg=red}
		\begin{center}
			\vskip5mm
  			{\large AVAZ for autism}
  			\vskip5mm
  			\vskip5mm  			
		\end{center}
	\end{variableblock}
\end{frame}


\section{Inventors}
\begin{frame}
	\frametitle{Inventors: Invention Labs}
	\begin{center}
		\begin{figure}[invention]
			\includegraphics[scale=0.16]{images/Invention-Labs-Team.jpg} 
		\end{figure}
	\end{center}
\end{frame}

\begin{frame}
	\frametitle{About Inventors}
	\begin{itemize}
		\item Invention Labs is a start-up based on Chennai and incubated at IIT Madras.
		\item Voted one of the hottest start-ups in India by Business Today in 2009.
		\item Invention Labs was founded by an alumni of IIT Madras.
	\end{itemize}
\end{frame}

\begin{frame}
	\frametitle{What is AVAZ?}
	\begin{itemize}
		\item AVAZ is an Augmentative and Alternative Communication (AAC) device for children with Cerebral Palsy.
		\item AVAZ is a portable speech synthesizer which can be controlled by gross motor movements of a child.
		\item The device helps speech-impaired children to become much more independent and free from their existing barriers.
		\item AVAZ is also portable, allowing the child to carry it around and even mount it on a wheelchair.
	\end{itemize}
\end{frame}

\begin{frame}
	\frametitle {Motivation of the device}
	\begin{itemize}
		\item Provide a \textcolor{red}{voice} to the non-verbal, allowing autistic children to communicate and express themselves
		\item For uplifting the autistic children.
		\item Enable speech-impaired children to communicate easily 
	\end{itemize}
\end{frame}

\section{Time line}
\begin{frame}
	\frametitle{Time line of AVAZ}
	\tlcventry{2008}{2011}{AVAZ time line}{}{}{}{}
	\tldatecventry[brown]{2008}{AVAZ }{Invention Labs came up with idea}{}{}{}{}
	\tldatecventry[magenta]{2009}{AVAZ }{received 10 Lakh rupees grand from the Department of Scientific and Industrial Research}{Government of India}{}{}
	\tldatecventry[blue!70!black]{2010}{AVAZ}{beta version released}{}{}{}
	\tldatecventry[red]{2011}{AVAZ app version 1.0}{released on 18 May 2011}{}{}{}
\end{frame}

\section{Commercialized}
\begin{frame}
	\frametitle {AVAZ is commercialized!}
	\begin{itemize}
		\item AVAZ device is \textcolor{red}{commercialized} - \$800 per piece.
		\item AVAZ app is available for iPad and Android.
		\item Invention Labs \textcolor{red}{earns} \$99 upon one app download.
	\end{itemize}
\end{frame}

\section{Opinion on IPR}
\begin{frame}
	\frametitle {Intellectual Property Rights}
	\begin{itemize}
		\item Existing autism device costs around \$10,000 -- \$12,000. Where as AVAZ device costs only \$800. 
		\item As other autism device manufactures have \textcolor{red}{not} enforced IPR, AVAZ became popular in the western countries and in India. 
		\item Decline to enforce the intellectual property rights by other device manufactures has strengthened AVAZ's market position.
	\end{itemize}
\end{frame}

\frame{
  	\vspace{2cm}
	\begin{center}
  		{\huge Thank you!}
	\end{center}
  	\vspace{1cm}
  	\begin{flushright}
        Seshagiri Prabhu \\
    	\structure{\footnotesize{seshagiriprabhu@gmail.com} \\ AM.EN.P2CSN12028 \\  \href{https://github.com/seshagiriprabhu/software-protection-lab}{Assignment repo}}
  	\end{flushright}
}

\end{document}